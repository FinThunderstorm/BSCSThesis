% !TEX root = ./HY-CS-main.tex
\chapter{Yhteenveto\label{yhteenveto}}

Oppimisanalytiikkaa voidaan kuvata oppijoista kerättävän tietoaineiston mittaamisena, keräämisenä, analysointina ja raportointina \citep{siemensLearningAnalyticsEmergence2013,clowLearningAnalyticsCycle2012}. Tätä hyödynnetään oppimisen ja sen ympäristön ymmärtämiseen sekä optimoimiseen.

Oppimisanalytiikkaa voidaan yleensä kuvata neljän osa-alueen syklinä: oppija, data, analyysi ja toiminta \citep{clowLearningAnalyticsCycle2012}. Oppimisanalytiikan lähtökohta on oppija, joka tuottaa oppimisanalytiikassa käytettävän datan \citep{wolffImprovingRetentionPredicting2013}. Kerättävä data voi olla esimerkiksi oppimisympäristöistä. Kerätyn datan pohjalta voidaan tarkastella ja analysoida oppimisprosessin havainnollistamiseksi, joka on oppimisanalytiikan tärkein vaihe. Toiminnan avulla on tarkoitus vaikuttaa oppijaan ja tarjota mahdollisuuksia kehittää omaa oppimistaan.

Oppimisanalytiikassa tuloksien hyödyntäjiä voi olla useita, eikä toiminta vältämättä tavoita aina oppijaa \citep{clowLearningAnalyticsCycle2012}. Oppimisanalytiikasta saatavia tietoja voidaan hyödyntää oppijan toiminnan kehittämisen lisäksi opettajan toiminnan kehittämiseen, hallintohenkilöstötasolla esimerkiksi kohdistettaessa resursseja tai vielä laajemmalla tasolla esimerkiksi osana opetussuunnitelmatyötä \citep{clowOverviewLearningAnalytics2013}. Tällöin voidaan havaita, että oppimisanalytiikka voi vaikuttaa sekä yhden tietyn oppijan toimintaan, mutta laajemmin myös useiden oppijoiden toiminnan kehittämiseen. Kehittämällä raportointia kohti enemmän datalla johtamista, pystytään tukemaan oppijoita oppimisanalytiikalla saavuttamaan tavoitteitaan paremmin.

Tilastollisten mallien sekä ennustavan mallintamisen avulla voidaan oppimisanalytiikan avulla tarjota ohjausta oppijoiden oppimishaasteisiin sekä tarjota kohdistettua tukea tietoaineiston avulla \citep{ranjeethSurveyPredictiveModels2020}. Eri oppijan tietoa sisältävistä järjestelmistä, kuten Moodelsta saatavaa tietoaineistoa voidaan käsitellä eri datanlouhinta-, koneoppimis- ja keinoälymenetelmillä.

Yksi esimerkki oppimisanalytiikan tietoaineiston lähteeksi on Moodle. Moodle on avoimen lähdekoodin verkko-oppimisympäristö, jolla on yli 315 miljoonaa käyttäjää eri puolilla maailmaa \citep{dougiamasPowerOpenEducational2021,dougiamasMoodle2022,moodle.orgMoodleStatistics}. Moodle tarjoaa useita erilaisia aktiviteettejä, jotka tallentavat tietoa oppijan toiminnasta relaatiotietokantaan. Oppimisanalytiikan osalta tentti, palaute, työpaja, oppitunti, keskustelualue ja H5P tuottaa konkreettista tietoa oppijan suoritumisesta. Moodlen tapahtumalokiin Event API:n kautta tallennetut tapahtumat tuottavat oppijan toiminnan kannalta merkittävää tietoaineistoa esimerkiksi tehtävien avauskertojen sekä oppijan toiminnan ajoittumisen osalta \citep{dougiamasLoggingMoodleDocs2021, abdullahLearningStyleClassification2015}.

Järjestelmistä saatavaa tietoaineistoa voidaan käsitellä ennustaviin malleihin pohjautuvilla luokittimilla \citep{hamalainenClassifiersEducationalData2010}. Bayesin teoreemalla voidaan laskea todennäköisyystapahtumien avulla kunika todennäköisesti jokin asia A tapahtuu, mikäli ehto B saa tietyn arvon \citep{natinggaDataScienceAlgorithms2018}. Toinen mahdollinen ennustava malli on käyttää regressioanalyysiä, kuten lineaarista regressiota \citep{rossIntroductoryStatistics2017}. Se kuvaa yhden selitettävän ja yhden tai useamman selittävän muuttujan välistä yhteyttä toisiinsa.

Valmisteltaessa tietomallia on huomioitava myös aineiston käsitteleminen ennen sen syöttämistä ennustavalle mallille \citep{romeroSurveyPreProcessingEducational2014, rossIntroductoryStatistics2017}. Esikäsittelyssä kerätään kaikki tarvittava tietoaineisto ja tämän jälkeen tietoaineistoa ryhmitellään, siistitään ja muokataan analyysiin sopivaan muotoon. Lisäksi tarkastellaan mahdollisten tekomuuttujien tarve.

Oppimisanalytiikan avulla voidaan kohdentaa ehdotuksia yksittäisiin oppijoihin esimerkiksi tunnistamalla mahdollisesti kurssin reputtavia oppijoita \citep{barberCourseCorrectionUsing2012} sekä etsimällä relaatioita oppijoiden verkko-oppmisympäristön toiminnan ja akateemisen menestyksen väliltä \citep{agudo-peregrinaCanWePredict2014}. Myös koko koulupolkua voidaan tarkastella ennustamalla valmistuuko koulutukseen hakija tavoiteaikataulussa \citep{barberCourseCorrectionUsing2012a}.

Opetuksen kehittämiseen voidaan oppimisanalytiikkaa soveltaa esimerkiksi kehittämällä resurssien sijoittelua ja käyttöä \citep{longPenetratingFogAnalytics2011, romeroEducationalDataMining2010}. Käytännössä voidaan hyödyntää esimerkiksi kurssien heikkojen kohtien, oppilaitoksen onnistumisien ja haasteiden löytämiseen sekä opetussuunnitelmien sekä hallinnon päätöksien tukemiseen. Näiden avulla voidaan kohdistaa resurssit vastaamaan todellista tarvetta sekä tarkastelemaan onko jollekin osa-alueelle kohdistettava enemmän opetusresursseja.

Oppimisanalytiikalla ei voida korvata oppijoiden ohjausta, vaan oppimisanalytiikka on yksi työkalu kaikkien muiden työkalujen joukossa \citep{auvinenOppimisanalytiikkaTuleeOletko2017}. Oppimisanalytiikalla voidaan tehostaa tätä toimintaa. Se tulee ymmärtää työvälineenä, jonka avulla ymmärretään paremmin oppimista ilmiönä ja havainnollistaa mitä oppimisessa tapahtuu. Oppijan on helpompi ymmärtää oppimisessa olevia haasteita, kun ne pystytään esittämään paljon konkreettisemmin.
