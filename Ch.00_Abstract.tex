% !TEX root = ./HY-CS-main.tex
\begin{abstract}

Nykypäivänä opetuksen tukena oppilaitoksissa hyödynnetään useita erilaisia verkkopalveluita. Yhteistä jokaiselle näistä palveluista on, että ne tallentavat tietoa käyttäjän toiminnasta sekä käyttäjän suorituksien arvioinneista. Tämän tutkielman tavoitteena on perehtyä mahdollisuuksiin hyödyntää tätä tietoa oppijan oppimistuloksien sekä opetuksen parantamiseksi.

Tutkielmassa tarkastellaan oppimisen analysoimisen tarpeita ja miksi oppimisanalytiikkaa halutaan tehdä. Lisäksi tarkastellaan Moodlea datalähteenä oppimisanalytiikalle sekä tutustutaan tarkemmin kahteen erilaiseen oppimisanalytiikassa käytettyyn tietomalliin. Lopuksi tarkastellaan kuinka oppimisanalytiikan avulla voidaan tietolähteiden avulla muodostaa oppijaa sekä opetusta tukevia analyysejä.

\end{abstract}

\begin{otherlanguage}{english}
\begin{abstract}

Nowadays schools and universities are using many kinds of web services to support learning and teaching. These services have things in common: all of them are saving information about users’ interactions and performance assessments. This thesis’ plan is to dive into possibilities on how this data can be used to improve learners’ outcome and improve teaching to be better.

Thesis will take look into needs of learning analytics and the reasons why learning analytics is done. As an example, Moodle will be investigated as a data source and how two different kind of data model used in learning analytics are built. In the end there are some examples how learning analytics can be used with data sources to help learner and to make teaching supportive analytics.

\end{abstract}
\end{otherlanguage}
