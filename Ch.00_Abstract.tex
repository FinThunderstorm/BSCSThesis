% !TEX root = ./HY-CS-main.tex
\begin{abstract}

Nykypäivänä opetuksen tukena oppilaitoksissa hyödynnetään useita erilaisia verkkopalveluita. Yhteistä jokaiselle näistä palveluista on, että ne tallentavat tietoa käyttäjän toiminnasta sekä käyttäjän suorituksien arvioinneista. Tämän tutkielman tavoitteena on perehtyä mahdollisuuksiin hyödyntää tätä tietoa oppijan oppimistuloksien sekä opetuksen parantamiseksi.

Tutkielmassa tarkastellaan oppimisen analysoimisen tarpeita ja miksi oppimisanalytiikkaa halutaan tehdä. Lisäksi tarkastellaan Moodlea datalähteenä oppimisanalytiikalle sekä tutustutaan tarkemmin kahteen erilaiseen oppimisanalytiikassa käytettyyn tietomalliin. Lopuksi tarkastellaan kuinka oppimisanalytiikan avulla voidaan tietolähteiden avulla muodostaa oppijaa sekä opetusta tukevia analyysejä.

\end{abstract}

\begin{otherlanguage}{english}
\begin{abstract}

\end{abstract}
\end{otherlanguage}
