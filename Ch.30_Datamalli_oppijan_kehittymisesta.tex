\chapter{Datamalli oppijan kehittymisesta\label{datamallioppijankehittymisesta}}
\color{red}




% Mattsonin luentoselostuksesta FOR-007, perustuu Agrestin kirjaan kurssimateriaalina
% - Yleinen lineaarinen malli = normaalilineaarinen malli viitaten normaalijakaumaoletukseen
% - Yleistetyssä lineaarisessa mallissa voi olla myös muita jakaumia käytetty, kuten Poisson.


Ajatus siis, että luodaan malli opiskelijoiden menestymisestä \\
- > verrataan opiskelijan menestymistä yleiseen malliin \\
- > katsotaan onko opiskelija mallin mukaisella radalla \\
- > jos opiskelija tipahtaa, niin mallin avulla havaitaan tämä

\section{Yleistetty malli} % Agrestin kirjasta
Yleistetyt lineaariset mallit koostuvat kolmesta osasta: satunnaisosasta, systemaattisesta osasta ja linkkifunktiosta \cite{agrestiFoundationsLinearGeneralized2015}.

Matemaattiselta taustaltaan oppimisanalytiikassakin halutaan muodostaa tilastollinen malli, jolla halutaan tutkia tekijöiden $x$ vaikutusta seuraukseen $y$ \cite{mottonenYleistetytLineaarisetMallit2021}. 

% Table 1.1 viitattuna kivasti
% Yleistettyjä lineaarisia malleja ovat esimerkiksi regressioanalyysi, varianssianalyysi, 

% Lineaariset mallit \& least squares

Selittäviä tekijöitä on kolmenlaisia: sekoittavia tekijöitä, väliintulevia tekijöitä ja riippumattomia syytekijöitä. \cite{mottonenYleistetytLineaarisetMallit2021} (YML-04).

Jos esimerkiksi meillä on lineaarinen malli $$\mu = \beta_0 + \beta_1x_{i1} + \beta_2x_{i2}$$, jossa selittävinä muuttujina kurssiolosuhteista $x_{i1}$ esittää opiskelijan tehtäväpisteitä ja $x_{i2}$ esittää opiskelijan tenttiarvosanaa. Selitettävänä muuttujana $y_i$ on opiskelijan kurssiarvosana. 

Näistä selittävistä muuttujista materiaalin avauskerrat ja käytetty aika ovat sekoittavia tekijöitä. Arvosana tehtävästä puolestaan on väliintuleva muuttuja.

% loput taustotuksesta ja muuttujien sekä osien selityksistä

% lineaarisen mallin sovittaminen
Multiple linear regressio \cite{agudo-peregrinaCanWePredict2014} \\
Linear regression \cite{tempelaarSearchMostInformative2015} \\
Logistic regression \cite{barberCourseCorrectionUsing2012} \\
Logistic regression \cite{garmanLogisticApproachPredicting2010} \\
Naïve Bayes algorithm \cite{barberCourseCorrectionUsing2012} \\


Linear regressionista hyvää tietoa \cite{rossIntroductoryStatistics2017} - Ch 12 \\
Naive Bayessian Ch 16


Ennen datan syöttämistä millekään analysiontia tai luokittelua tekevälle algoritmille tai mallille, tarvitsee sitä mahdollisesti käsitellä, ns. pre-processing.

\color{black}
\color{green}
Oppimisanalytiikassa opiskelijan suoriutumista halutaan luokitella (HOX! Tarkasta ja täydennä muotoilu) \cite{akcapinarUsingLearningAnalytics2019}. Useissa oppimisanalytiikkaa käsitelleissä tutkimuksissa on kokeiltu erilaisia luokittelualgoritmejä data-aineistolle löytääkseen parhaiten toimivan mallin. Kokeiltuja algoritmejä ovat Naive Bayes, Classification Tree, Random Forest, Support Vector Machines, Neural Network, CN2 rules ja k-nearest neighbour. Parhaiten toimivaa mallia voidaan etsiä esimerkiksi suorituskykymittauksin, jossa tarkastellaan Accuracy, Sensitivity, Specificity ja F-Measure. 

Yksi tapa toteuttaa malli on käyttää ristivalidointia \cite{deisenrothMathematicsMachineLearning}. Yksi tälläinen ristivalidoinnin malli on k-fold cross validation. Tässä mallissa aineisto jaetaan $k$ osaan, joista yhtä osaa käytetään testiaineistona $\mathcal{V}$ ja $k-1$ osaa koulutusaineistona $\mathcal{R}$. Tällöin käytettävästä koulutusaineistosta käytetään suurin osa mallin kouluttamiseen, mutta samasta aineistosta saadaan myös tarkastusaineisto. Ristiinvalidoinnissa käydään läpi kaikki mahdolliset $k$ vaihtoehtoa valita testausaineisto eli jaetaan data-aineisto kahteen osaan $D = \mathcal{R} \cup \mathcal{V}$, missä $\mathcal{R} \cap \mathcal{V} = \emptyset$. Näiden k-suorituskerran muodostamien mallien suorituskyky tarkastellaan keskiarvona.

% Lisää tähän kiva taulukkokaaviohimmeli k-fold-cross-validationista, Figure 8.4 %

Mallin kouluttamisen jälkeen koulutusaineistolla $\mathcal{R}$ tarkastellaan koulutetun mallin $f$ suorituskykyä testausaineiston $\mathcal{V}$ avulla. Tälle halutaan laskea root mean square error RMSE. Näin toteutettaessa saadaan jokaiselle $k$-osalle toteutettua malli $f^{(k)}$ koulutusaineiston $\mathcal{R}^{(k)}$ avulla. Näiden pohjalta voidaan laskea tarkastusaineiston $\mathcal{V}^{(k)}$ avulla empirical risk $R(f^{(k)}, \mathcal{V}^{(k)})$. Tämän avulla ristiinvalidointi pystyy arvioimaan expected generalization error $$\mathds{E}_{\mathcal{V}}[R(f,\mathcal{V})] \approx \frac{1}{K}\Sigma^{K}_{k=1} R(f^{(k)}, \mathcal{V}^{(k)}).$$ Prosessissa käytettävässä arvioinnissa on kaksi lähdettä, joista toinen on ettei rajatulla koulutusaineistolla välttämättä saada parasta mahdollista $f^{(k)}$ ja toinen on, ettei testausaineistolla saada tarkkaa arviota riskistä $R(f^{(k)}, \mathcal{V}^{(k)})$.
\color{black}




\color{red}

Toinen mahdollisuus tehdä tilastollista analyysia kerätylle oppimisdatalle on regressioanalyysi. Regressioanalyysi voidaan tehdä usealla eri tavalla. Tälläisiä tapoja on esimerkiksi yksinkertainen lineaarinen regressio, usean selittäjän lineaarinen regressio ja logistinen regressio. Data-aineiston pohjalta muodostettavalla regressiomallilla voidaan ennustaa esimerkiksi kuinka opiskelija tulee pärjäämään kurssilla.

Lineaarinen regressio kuvaa selittävän muuttujan ja selitettävän muuttujan yhteyttä toisiinsa \cite{rossIntroductoryStatistics2017}. Yksinkertaisessa lineaarisessa regressiosa selittäviä ja selitettäviä muuttujia on molempia yksi. Yksinkertainen lineaarinen regressio voidaan esittää kaavana $$Y = \alpha + \beta x + e,$$ jossa $x$ kuvaa selittävää muuttujaa ja $y$ kuvaa selitettävää muuttujaa. Parametrit $\alpha$ ja $\beta$ kuvaa... Muuttuja $e$ kuvaa satunnaista virhettä \dots, jonka *mean* keskiarvo on nolla. HOX! OLS \& GLS

Parametrien $\alpha$ ja $\beta$ estimointi voidaan tehdä pienimmän neliösumman estimoinnilla (least square estimator).

Esimerkiksi yksinkertaisesta lineaarisesta regressiosta on opiskelijan kolmannen viikon tehtävän tulos vaikuttaa opiskelijan kurssiarvosanaan. (etso parempi)



Usean selittäjän lineaarinen regressio kuvaa kuinka useampi selittävä muuttuja vaikuttaa selitettävään muuttujaan \cite{rossIntroductoryStatistics2017}. Matemaattisena kaavana esitettynä tämä olisi $$Y = \beta_0 + \beta_1x_1 + \beta_2x_2 + \ldots + \beta_kx_k,$$ jossa $Y$ on selitettävä muuttuja, ja $x_i$ kuvaa selittäviä muuttujia, jossa $i = 1, \cdots, k$. Regressioparametrejä yhtälössä kuvaa $\beta_0, \beta_1, \cdots, \beta_k$ ja satunnaisvirhettä $e$.

HOX! Multikolineaarisuus vaara

Logistinen regressio kuvaa $\cdots$

Dummy-muuttuja






% Selitettävä muuttuja = opiskelijan menestys kurssilla = response variable
% Selittävinä muuttujina = opiskelijan toiminta Moodlessa. = input variable

\section{Yksitäiseen oppijaan kohdennetut ehdotukset}

\begin{enumerate}
    \item arviointi- ja muiden seuraamisperiaatteiden muotoileminen malliksi
    \item kurssiarvosanan ennustaminen kurssin edistyessä
    \item suositeltavat jatkokurssit
    \item Educational Data Mining EMD, ainakin \cite{romeroEducationalDataMining2010}
    \item etiikka?! \cite{kailaEthicalConsiderationsLearning2019}
\end{enumerate}

\section{Ehdotuksien tulkinnan rajoitteet}  

\begin{enumerate}
    \item virhearviot
    \item yhden asian tajuamatta jääminen !== huono kurssimenestys
    \item model bias
\end{enumerate}


tilastollinen malli kuvaa optimia, ja verrataan kuinka data sopii tähän malliin