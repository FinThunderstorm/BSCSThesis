% !TEX root = ./HY-CS-main.tex
\chapter{Datamallin hyödyntäminen oppimisanalytiikassa\label{hyodyntaminen}}

Oppimisanalytiikan hyödyntäjiä on useissa käyttäjäryhmissä, kuten oppijat, opettajat, kurssikehittäjät(opintoesimies tms), organisaatiot ja hallintohenkilöstö \citep{romeroEducationalDataMining2010}. Käyttäjäryhmät hyödyntävät

\color{red} Alusta kertomalla oppimisanalytiikan useammasta hyödyntämiskerroksesta, oppija, opettaja, koulu, valtakunnallinen. Oppimisanalytiikan avulla voidaan tehdä opetuksen kehttämistä \citep{romeroEducationalDataMining2010}. Table 1. \color{black}

Ennustavien mallien avulla muodostetaan keskimääräistä oppijaa kuvaavia malleja, joiden avulla yksittäisiä oppijoita voidaan vertailla \citep{wolffImprovingRetentionPredicting2013}. Ennustavan mallinnuksen avulla voidaan ennustaa esimerkiksi kuinka oppija tulee menestymään kurssilla ja onko oppija pääsemässä kurssia läpi. Tämä tapahtuu vertailemalla oppijaa muodostettuun malliin ja ennusteen perusteella katsotaan onko oppija vaarassa olla läpäisemättä kurssia.

\section{Yksitäiseen oppijaan kohdennetut ehdotukset}

\color{blue}
Oppijan menestymistä kurssilla voidaan ennustaa eri tarkoituksiin \citep{barberCourseCorrectionUsing2012a}. Tähän voidaan hyödyntää aiempien kurssien menestystietoa muista lähteistä, sekä kurssin edistyessä lisätä kurssisuorituksista saatavaa tietoa mukaan analyysiin. Yhdistämällä tämän visualisointiin, voidaan tarjota oppijalle reaaliaikainen näkymä kurssimenestyksestä ja hyödyntää tätä motivaation lähteenä. Kurssitasolla voidaan seurata opiskelijan toimintaa kurssilla ja tehdä havaintoja kurssin edistymisestä ja menestymisestä \citep{longPenetratingFogAnalytics2011,siemensLearningAnalyticsEmergence2013}. Tätä voidaan tehdä esimerkiksi luokittelulla tai ennustavilla malleilla.

Naiivilla Bayesin luokittimella voidaan esimerkiksi yrittää tunnistaa opiskelijoita, jotka ovat vaarassa saada hylätyn osallistumaltaan kurssilta \citep{barberCourseCorrectionUsing2012}. Selittävinä muuttujina oli henkilöön liittyviä taustatietoja, suoritettujen opintopisteiden suhde yritettyihin opintopisteisiin sekä toimintaa verkko-oppimisympäristön keskustelualueella. Selittäville muuttujille oli annettu eri painoarvoja riippuen kurssin viikosta.

Usean selittäjän lineaarista regressiota voidaan hyödyntää esimerkiksi etsittäessä eri relaatioita opiskelijan verkko-oppimisympäristön toiminnan ja akateemisen menestyksen väliltä \citep{agudo-peregrinaCanWePredict2014}. Riippumattomia selittäviä muuttujia olivat eri tyyppiset interaktiot verkko-oppimisympäristössä ja riippuvana selittävänä muuttujana jokaisen oppijan saamana kurssin päättöarvosanana esittety akateeminen menestys, joiden väliltä löydettiin merkittäviä relaatioita.

Yksi oppijan menestymiseen perustuva tarkastelu on onko oppija vaarassa pudota kurssilta \citep{oliveSupervisedLearningFramework2018, suhonenUsingMoodleData2019}. Tarkastellaan oppijan toimintaa verkko-oppimisympäristössä ja yritetään löytää eri merkkejä oppijan putoamisesta kurssilta. Tarkastelua voidaan laajentaa eri kurssien väliseksi \citep{kinnari-korpelaOppimisanalytiikallaTehokkaampaanOhjaukseen2020} ja analytiikan löytäessä putoamisvaarassa olevan oppijan esimerkiksi oppijan jättäessä ilmoittautumatta kursseille ja lopulta hiljaisesti jättämällä opinnot kesken, voidaan hänelle tarjota kohdistetusti tukea oppimiseen jo aikaisessa vaiheessa.
% Etsi tämän lähde jostain: Paras tapa tunnistaa oppijan mahdollinen putoaminen on vertailla oppijan aiempaan käyttäytymiseen ja tunnistaa muutoksen merkkejä siinä (tälle oli kiva lähde, löytynee luvusta 3 :D).

Moodleen on sisäänrakennettu Learning Analytics API:n avulla opiskelijoiden tippumisen tunnistamisen tarjoava malli \citep{oliveSupervisedLearningFramework2018,monllaoAnalyticsAPIMoodleDocs2021}. \color{red} Avaa Moodle osuutta vielä tarkemmin. \color{black}

Yksi taso on hyödyntää oppimisanalytiikkaa sisällön suosittelemiseen \citep{longPenetratingFogAnalytics2011,siemensLearningAnalyticsEmergence2013}. Tässä oppijan oppimispolku muotoillaan osaamista vastaavaksi esimerkiksi ohjaamalla perusasiat jo hyvin osaava oppija haasteellisemmalle kurssille tai tarjotaan heikommin pärjäävälle opiskelijalle taitotasoa vastaavia tehtäviä. \color{red}(HOX! Tsiikaa noi muut kolme muuta Longin nostoa sekä table 1)\color{black}

Oppimispolku on ohjeistus, joka kertoo oppimistehtävien ohjeistukset ja tavoitteet, sekä havainnollistaa oppimisen edistymistä kurssin aikana \citep{toivolaFlippedLearningKaanteinen2017}. Oppimispolun halutaan mahdollistaa oppijan oman luontaisen oppimistahdin hyödyntäminen. Oppimisanalytiikan avulla voidaan visualisoida oppijan edistyminen oppimispolulla ja tarjota myös suosituksia seuraavista tehtävistä \citep{longPenetratingFogAnalytics2011}. Jos oppija ei ole vielä ymmärtänyt jotain oppimispolun osa-aluetta, voi analytiikka ehdottaa lisätehtävää osaamisen vahvistamiseksi ennen seuraavaan osa-alueeseen siirtymistä. Pikapalautetta voidaan hyödyntää itsearvioiden toteuttamiseen ja edelleen analytiikan tukena.


\section{Opetuksen kehittämiseen kohdennetut ehdotukset}

\color{blue}
Oppimisanalytiikasta saatavalla datalla voidaan kehittää resurssien sijoittelua ja käyttöä \citep{longPenetratingFogAnalytics2011}. Oppimisanalytiikalla voidaan löytää nykyisistä kursseista heikkoja kohtia, joihin ratkaisu voi olla esimerkiksi uuden kurssin luominen tai nykyisen kehittäminen tukemaan osaamisvajeen paikkaamista. Tämä voi näkyä esimerkiksi oppimateriaalin kehittämisenä, mikäli analytiikka osoittaa tietyn osa-alueen tehtävien menevän muita heikommin, kun saman aikaisesti tiettyä opetusmateriaalin osaa tarkastellaan muita enemmän.

Oppimisanalytiikan avulla voidaan ymmärtää paremmin oppilaitoksen onnistumisia ja haasteita \citep{longPenetratingFogAnalytics2011}. Oppilaitoksen tuottavuutta ja tehokkuutta voidaan kehittää hyödyntämällä viimeisintä tietoa ja haasteisiin pystytään vastamaan nopeasti.

Oppimisanalytiikan avulla voidaan kehittää opetussuunnitelmia sekä tukea hallinnon päätöksiä \citep{romeroEducationalDataMining2010,longPenetratingFogAnalytics2011}

Moodleen on sisäänrakennettu Learning Analytics API:n avulla kurssin opetuksen puuttumisen tunnistava malli \citep{monllaoAnalyticsAPIMoodleDocs2021}. \color{red} Avaa Moodle osuutta vielä tarkemmin. \color{black}

Yksi ennustamisen mahdollisuus on tarkastella valmistuuko koulutukseen hakija ennusteen mukaan tavoiteaikataulussa \citep{barberCourseCorrectionUsing2012a}.

Opetuksen tutkijat \citep{romeroEducationalDataMining2010}

\color{black}

\color{red}
\section{Ehdotuksien tulkinnan rajoitteet}

\begin{enumerate}
    \item virhearviot ja model bias
    \item mallien ennustuksien paikkaapitävyyden todennäköisyydet - kuinka todennäköisesti ennustuksen tulos pitää paikkansa. voidaanko 72 prosentin todennäköisyyttä pitää sellaisena, että se toimii luotettavana ohjauksen työkaluna?
\end{enumerate}