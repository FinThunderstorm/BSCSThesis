\chapter{Oppimisen analysoinnin tarpeet\label{oppimisenanalysoinnintarpeet}}

Halutaan tutkia opiskelijan kurssisuorituksien vaikutusta opiskelijan menestykseen kurssilla.

\section{Oppimisanalytiikka pedagogisena työkaluna}

Data voidaan jaotella kahteen laatuun, aktiiviseen ja passiiviseen. \cite{maddenDigitalFootprints2007} \cite{mikkolaMitaOppimisanalytiikka2019}

Määritelmä \cite{siemensLearningAnalyticsEmergence2013}

Table 1 \cite{longPenetratingFogAnalytics}

AnalytiikkaÄly-hanke, tsekkaa tää -> datan käsittelystä tarkasta tukevien lähteiden varalta \cite{kokkonenEffectsDataCleaning}
E. Kaila (UTU/HY)

Romero ja Siemens toistuvia nimiä 

Learning analytics LA <--> Educational Data Mining EDM

Virtual Learning Environment VLE <--> Learning Management System LMS

\section{Moodle datalähteenä}

Mitä kaikkea saadaan ulos

SQL
\begin{enumerate}
    \item raa'alla vuomalla SQL hakuina kannasta => aktiviteettitunnisteet apuna?
\end{enumerate}


Moodlen LA API => onko tästä mihinkään? ainakin pikavilkaisu paikallaan

Erilaiset saatavat aineistot
\begin{enumerate}
    \item Gradebook
        \begin{itemize}
            \item quizzes
            \item assignments
            \item workshop etc
            \item forum
        \end{itemize}
    \item Events
        \begin{itemize}
            \item core course module viewed
            \item modassign submission created
            \item modassign submission updated
        \end{itemize}
    \item suorituskertojen määrä
    \item pikapalaute (block\_point of view)
    \item muu palaute
\end{enumerate}

\section{Päättelymahdollisuudet}

Oppimisanalytiikan kolmijako Daniel @ \cite{eoppimiskeskusOppimisanalytiikkaTuleeOletko2017} % sisältää myös hyviä lähteitä 

Mitä voidaan päätellä
\begin{enumerate}
    \item opiskelijan menestys kurssilla
    \item dropout alert
    \item opetusmenetelmien tehokkuus
    \item haasteet oppimateriaalissa - katseltu useasti ja tehtävä menee edelleen päin honkia
\end{enumerate}
