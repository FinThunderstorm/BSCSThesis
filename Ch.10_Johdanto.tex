% !TEX root = ./HY-CS-main.tex
\chapter{Johdanto\label{johdanto}}

Tutkimuskysymykset:
\begin{enumerate}
    \item Millaista dataa Moodlesta saadaan oppimisanalytiikan prosessin käyttöön
    \item Miten saatua dataa voidaan hyödyntää oppijan tukemiseksi oppimisanalytiikan avulla
\end{enumerate}

HOX! Punaisella merkityt tekstiosuudet ovat tutkimuspäiväkirjan sisältöä, johon olen kirjannut ylös erilaisia havaintoja ja hyviä lähdeaineistoja talteen hyödynnettäväksi myöhemmissä vaiheissa. Normaalit tekstiosuudet ovat varsinaista kandidaatin tutkielman sisältöä.

Selitettävää:
Learning analytics LA <--> Educational Data Mining EDM
Virtual Learning Environment VLE <--> Learning Management System LMS

LMS on on verkkopohjainen järjestelmä, joka mahdollistaa oppimateriaalin, opiskelijoiden toiminnan, tehtävätyökalujen ja oppijan edistymisen seurannan. \citep{mohdChoosingRightLearning2016}. \citep{romeroSurveyPreProcessingEducational2014}