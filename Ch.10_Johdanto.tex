% !TEX root = ./HY-CS-main.tex
\chapter{Johdanto\label{johdanto}}

Nykyisin oppilaitoksissa hyödynnetään useita erilaisia verkkopalveluita opetuksen tukena. Tälläisiä verkkopalveluita on esimerkiksi Moodle, ItsLearning tai Thinglink. Yhteistä jokaiselle näistä palveluista on, että ne tallentavat tietoa käyttäjän toiminnasta sekä käyttäjän suorituksien arvioinneista. Kuinka näiden järjestelmien keräämä tieto voidaan hyödyntää oppijan avuksi? Tämän tutkielman tavoitteena on perehtyä mahdollisuuksiin hyödyntää tätä tietoa oppijan oppimistuloksien parantamiseksi.

Tahtotilana on muodostaa käsitys siitä, kuinka oppijalle voidaan esittää hänen oma toimintansa oppimisen eteen selkeässä helposti ymmärrettävässä muodossa. Tämän tarkoituksena on tarjota mahdollisuus ymmärtää omaa toimintaa sekä antaa mahdollisuuksia kehittää omaa toimintaa sen seurauksena. Päästäksemme tähän halutaan löytää oppijan käyttäytymisestä erilaisia malleja, joiden avulla pystyttäisiin havaitsemaan esimerkiksi oppijan tarvitseman tuki tai löytämään esimerkiksi isommassa kuvassa tarve kohdistaa lisää opetusta osaamisvajeen paikkaamiseksi.

Tutkielman toisessa luvussa tarkastellaan oppimisen analysoimisen tarpeita ja miksi oppimisanalytiikkaa halutaan tehdä. Luvussa käsitellään oppimisanalytiikan rakennetta sekä kuinka oppimisanalytiikka toteutuu käytännössä. Muodostetaan käsitys kuinka oppimisanalytiikka muodostaa syklin, joka tuottaa koko ajan näkymän oppimiseen.

Kolmannessa luvussa tarkastellaan Moodlea datalähteenä oppimisanalytiikalle sekä tutustutaan tarkemmin kahteen erilaiseen oppimisanalytiikassa käytettyyn datamalliin. Moodlen osalta nostetaan esille aktiviteettien merkitys datalähteenä analyysimalleille. Luvussa myös tarkastellaan Naiivin Bayesin mallin sekä regressioanalyysin toteuttamista.

Neljännessä luvussa tarkastellaan kuinka oppimisanalytiikan avulla voidaan datalähteiden avulla muodostaa oppijaa sekä opetusta tukevia analyysejä. Lisäksi tarkastellaan oppimisanalytiikan tulkinnan rajoitteita. Luvussa tuodaan esille kuinka oppija, opettaja sekä hallintohenkilöstö pystyy hyötymään oppimisanalytiikasta saatavasta analyyista osana heidän tekemistä.

\section{Terminologia}

Learning analytics LA <--> Educational Data Mining EMD \citep{romeroEducationalDataMining2010}
Virtual Learning Environment VLE <--> Learning Management System LMS

LMS on on verkkopohjainen järjestelmä, joka mahdollistaa oppimateriaalin, opiskelijoiden toiminnan, tehtävätyökalujen ja oppijan edistymisen seurannan. \citep{mohdChoosingRightLearning2016}. \citep{romeroSurveyPreProcessingEducational2014}
\color{black}

\section{Tutkimuskysymykset}
Tämä tutkielma antaa perusteet ymmärtää mitä oppimisanalytiikka on ja miten sitä voitaisiin hyödyntää oppijan tukena. Tutkielmassa on tarkasteltu mahdollisuuksia hyödyntää oppimisanalytiikkaa Moodle -verkko-oppimisympäristössä. Lisäksi tutkielma käsittelee oppimisanalytiikassa käytetyistä ennustavista malleista kahta erilaista, Bayesin mallia sekä lineaarista regressiota.

Tutkielma pyrkii vastaamaan tutkimuskysymyksiin "Miten oppimisympäristöstä saatua dataa voidaan hyödyntää oppijan tukemiseksi oppimisanalytiikan avulla" sekä "Millaista dataa Moodlesta saadaan oppimisanalytiikan prosessin käyttöön?" Ensimmäinen tutkimuskysymys käsittelee yleisesti oppimisanalytiikkaa ilmiönä ja kuinka sitä voidaan hyödyntää työkaluna. Toinen tutkimuskysymys pureutuu tarkemmin Moodleen lähdejärjestelmänä ja siihen, millaista dataa sieltä voitaisiin saada analysoitavaksi.
