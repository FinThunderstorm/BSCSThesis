% !TEX root = ./HY-CS-main.tex
\chapter{Johdanto\label{johdanto}}

Nykypäivänä opetuksen tukena oppilaitoksissa hyödynnetään useita erilaisia verkkopalveluita. Tälläisiä verkkopalveluita on esimerkiksi Moodle, ItsLearning tai Thinglink. Yhteistä jokaiselle näistä palveluista on, että ne tallentavat tietoa käyttäjän toiminnasta sekä hänen tekemien suorituksien arvioinneista. Kuinka näiden järjestelmien keräämä tietoaineisto voidaan hyödyntää oppijan avuksi? Tämän tutkielman tavoitteena on perehtyä mahdollisuuksiin hyödyntää tätä tietoa oppijan oppimistuloksien parantamiseksi.

Tahtotilana on muodostaa käsitys siitä, kuinka oppijalle voidaan esittää hänen oma toimintansa oppimisen eteen selkeässä helposti ymmärrettävässä muodossa. Tämän tarkoituksena on tarjota mahdollisuus ymmärtää omaa toimintaa sekä antaa mahdollisuuksia kehittää omaa toimintaa sen seurauksena. Päästäksemme tähän halutaan löytää oppijan käyttäytymisestä erilaisia malleja, joiden avulla pystyttäisiin havaitsemaan esimerkiksi oppijan tarvitseman tuki tai löytämään esimerkiksi isommassa kuvassa tarve kohdistaa lisää opetusta osaamisvajeen paikkaamiseksi.

Tutkielman toisessa luvussa \emph{\nameref{luku2}} tarkastellaan oppimisen analysoimisen tarpeita ja miksi oppimisanalytiikkaa halutaan tehdä. Luvussa käsitellään oppimisanalytiikan rakennetta sekä kuinka oppimisanalytiikka toteutuu käytännössä. Muodostetaan käsitys kuinka oppimisanalytiikka muodostaa syklin, joka tuottaa koko ajan näkymän oppimiseen.

Kolmannessa luvussa \emph{\nameref{luku3}} tarkastellaan Moodlea datalähteenä oppimisanalytiikalle sekä tutustutaan tarkemmin kahteen erilaiseen oppimisanalytiikassa käytettyyn tietomalliin. Moodlen osalta nostetaan esille aktiviteettien merkitys datalähteenä analyysimalleille. Luvussa myös tarkastellaan Naiivin Bayesin mallin sekä regressioanalyysin toteuttamista.

Neljännessä luvussa \emph{\nameref{luku4}} tarkastellaan kuinka oppimisanalytiikan avulla voidaan datalähteiden avulla muodostaa oppijaa sekä opetusta tukevia analyysejä. Luvussa tuodaan esille kuinka oppija, opettaja sekä hallintohenkilöstö pystyy hyötymään oppimisanalytiikasta saatavasta analyyista osana heidän tekemistä.

\section{Tärkeimmät termit}
Oppimisanalytiikka (Learning Analytics, LA) on tutkimusala, joka tutkii oppimista ja opettamista oppimisympäristöissä \citep{longPenetratingFogAnalytics2011}. Sen juuret ovat ihmisten ja koulutusjärjestelmän välisen vuorovaikutuksen analysoimisessa \citep{siemensLearningAnalyticsEmergence2013}. Oppimisanalytiikka jakautuu kahteen pääalueeseen, jotka ovat menetelmät ja sovellukset. Menetelmiin kuuluvat algoritmit ja tilastolliset mallit, joiden avulla analysoidaan kerättyä oppimisdataa. Sovelluksiin puolestaan kuuluu tavat, joilla menetelmien tuloksien avulla vaikutetaan ja kehitetään oppimistuloksia.

Oppimistiedon louhinta (Educational Data Mining, EDM) hyödyntää tilasto-, koneoppimis- ja datalouhintaalgoritmejä oppimisdatan käsittelyyn \citep{romeroEducationalDataMining2010,siemensLearningAnalyticsEmergence2013}. EDM kehittää erityisesti oppimisdatalle soveltuvia menetelmiä ja tähtää erityisesti oppimisdatalle yksilöllisten datapisteiden hyödyntämiseen. Sekä oppimisanalytiikka että EDM ovat osittain päällekäisiä tutkimusaloja etenkin menetelmien osalta, mutta eroja on löydettävissä kokonaiskuvassa. EDM keskittyy enemmän menetelmien kehittämiseen, kun taas oppimisanalytiikka puolestaan enemmän analyyttiseen lopputulokseen löytää järkeä sekä saada aikaiseksi toimintaa.

Verkko-oppimisympäristö (Learning Management System, LMS) on verkkopohjainen järjestelmä, joka mahdollistaa oppimateriaalin, opiskelijoiden toiminnan, tehtävätyökalujen ja oppijan edistymisen seurannan \citep{mohdChoosingRightLearning2016} missä vain oppijan haluamalla päätelaitteella. Verkko-oppimisympäristöjen juuret on oppijoiden suorituksien taltioinnissa ja pitää yllä kurssille ilmoittautuneista oppijoista, mutta nykypäivänä järjestelmät mahdollistavat verkkopohjaisen oppimisen kokonaisuudessaan sekä pääsyn keskitetysti kurssin tietoihin. Verkko-oppimisympäristön käytön aikana muodostuu tietoa suorituksista ja käyttäytymisestä taustalla on tiedon tallentamiseen käytetty tietokanta, johon on ryhmitelty oppimisesta kerättyä tietoa \citep{romeroSurveyPreProcessingEducational2014}.

\section{Tutkimuskysymykset}
Tämä tutkielma antaa perusteet ymmärtää mitä oppimisanalytiikka on ja miten sitä voitaisiin hyödyntää oppijan tukena. Tutkielmassa on tarkasteltu mahdollisuuksia hyödyntää oppimisanalytiikkaa Moodle -verkko-oppimisympäristössä. Lisäksi tutkielma käsittelee oppimisanalytiikassa käytetyistä ennustavista malleista kahta erilaista, Bayesin mallia sekä lineaarista regressiota.

Tutkielma pyrkii vastaamaan tutkimuskysymyksiin \emph{"Miten oppimisympäristöstä saatua tietoaineistoa voidaan hyödyntää oppijan tukemiseksi oppimisanalytiikan avulla"} sekä \emph{"Millaista tietoaineistoa Moodlesta saadaan oppimisanalytiikan prosessin käyttöön?"} Ensimmäinen tutkimuskysymys käsittelee yleisesti oppimisanalytiikkaa ilmiönä ja kuinka sitä voidaan hyödyntää työkaluna. Toinen tutkimuskysymys pureutuu tarkemmin Moodleen lähdejärjestelmänä ja siihen, millaista dataa sieltä voitaisiin saada analysoitavaksi. Tutkimuskysymyksien taustalla on työkokemuksen tuoma ymmärrys Moodlen toiminnasta ja siitä miten järjestelmä on rakentunut oppijan sekä opettajan näkökulmasta.
